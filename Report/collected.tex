\documentclass[12pt]{article} %scrartcl
%\usepackage{apacite}
\usepackage{wrapfig}
\setlength{\parindent}{0pt}
\usepackage{tgtermes}
\usepackage{setspace}
\usepackage{gensymb}
\usepackage{import}
\doublespacing
\usepackage{pdfpages}
\usepackage{graphicx}
\usepackage[subpreambles=true]{standalone}
\usepackage{float} %for the H (hammer) method
\usepackage[utf8]{inputenc}
\usepackage[backend=biber,style=apa,autocite=inline]{biblatex}
\DeclareLanguageMapping{english}{english-apa}
\addbibresource{../References.bib}
\usepackage{fancyhdr}
\pagestyle{fancy}
\lhead{Overall introduction}
\rhead{Victor M. Poulsen, Studie Nr.: 201707639}

\begin{document}

\title{Exam: Advanced Cognitive Neuroscience}
\author{Victor Møller Poulsen, Studie Nr.: 201707639}

\maketitle
\leavevmode
\section{Overall Introduction}

This collection of three neuroscience papers
focuses mainly on methodology and on
overarching theoretical frameworks. It is
generally less concerned with the specific
experimental findings than
it is with the methodology used to arrive
at those findings, and whether the methodology
warrants the inferences made. As such, specific
findings are brought in to the extent that they
ground the themes discussed, and ensure that the
discussion remains faithful to the best
experimental findings. Another common thread is
a focus on representation of information.
This is arguably a core question for neuroscience
\autocite{norman2006beyond}.
%However, it is a slippery concept, which is rarely
%defined clearly in the literature.
%By representation (e.g. of edge orientation), I
%will mean that the underlying neuronal signal
%discriminates between this class of information
%and other classes of information.

\vspace{3mm}

The first paper
focus on electrophysiology and brain oscillations.
Oscillations are generally thought to be
involved in routing and controlling flow
of information in the brain. A specific focus
is on the role of attention, since attention
necessarily is involved in selective routing
of information. The relation between the
concept of attention and neuronal realization
is as such a key relationship to specify for
any theory which attempts to explain the
role of neural oscillations.

\vspace{3mm}

Both the second and the third paper focus
on functional magnetic resonance imaging (fMRI).
The second paper discuss multivariate
pattern analysis (MVPA) as opposed to
classical univariate statistics in neuroscience.
A specific focus is on the advantages of linear and
non-linear classifiers in different contexts.
I will argue that for investigating representation
linear classifiers have certain advantages related
to interpretation. They are restrictive enough
to allow researchers to draw the inferences they
are often interested in from observed performance
of these statistical models.

\vspace{3mm}

The third paper focus mainly on convolutional
neural networks (CNNs) as models of the brain,
and specifically of the ventral visual system
of humans and non-human primates.
A model of a system must necessarily simplify
certain aspects of the system to be useful.
However, we do observe anomalous behaviour in
CNNs applied to visual object classification.
We will discuss which simplifications might
not be warranted in models of the brain,
but will also see that these models are useful
tools for understanding cognition while remaining
very different from the actual brain.

\includepdf[pages = -, scale = 1]{../paper1/paper1-final.pdf}
\includepdf[pages = -, scale = 1]{../paper2MVPA/paper2-v2.pdf}
\includepdf[pages = -, scale = 1]{../paper3/paper3-edit.pdf}
\includepdf[pages = -, scale = 1]{overall-discussion.pdf}

\printbibliography
\end{document}


