\documentclass[12pt]{article}
%\usepackage{apacite}
\usepackage{wrapfig}
\setlength{\parindent}{0pt}
\usepackage{tgtermes}
\usepackage{setspace}
\usepackage{gensymb}
\doublespacing
\usepackage{graphicx}
\usepackage{float}
\usepackage[utf8]{inputenc}
\usepackage[backend=biber,style=apa,autocite=inline]{biblatex}
\usepackage{fancyhdr}
\pagestyle{fancy}
\lhead{Paper 1}
\rhead{Victor M. Poulsen, Studie Nr.: 201707639}

\DeclareLanguageMapping{english}{english-apa}
\addbibresource{References.bib}
\setcounter{page}{1}

\title{Affective modulation of weighting function}
\author{Victor Møller Poulsen, Studie Nr.: 201707639}

\begin{document}
\maketitle
\leavevmode

\section{Description}
Prospect Theory \autocite{kahneman2013prospect,
tversky1992advances} (PT) is arguably the main model
of human decision making. It postulates a value
function $v$ for transforming objective value to
subjective utility, a probability weighting function $w$ to
transform probability into a subjective weight, and
an operation to combine these \autocite{
gonzalez1999shape}. There is evidence to support
the notion that the affect of outcomes modulates
the parameters of $w$ \autocite{
rottenstreich2001money,
hsee2004music} but a thorough investigation
of this effect is lacking. This study consits
of two sub-studies. In the first study subjects
will evaluate 10 items on a scale of affect.
In the second study subjects will indicate their
certainty equivalence (CE) as to gambles
involving these questions. Based on this, the
parameters of the weighting function

\[
	w(p) = \frac{\delta \cdot p^{\gamma}}
	{\delta \cdot p^{\gamma} + (1-p)^{\gamma}}
\]

are estimated for each of the 10 items, and it
is calculated whether level of affect (obtained in
study 1) modulates the parameters of the weighting
function. Note that the above is the two-parameter
weighting function
suggested in \autocite{gonzalez1999shape}.

\section{Hypotheses}

$H_1$: It is expected that the 10 questions in
study 1 will - on average - obtain significantly
different ratings as to affective quality.
This is necessary for the follow-up study to
make sense.

$H_2$ It is hypothesized that the $\gamma$
parameter will be higher for items that are
rated as being higher in affect.
(estimate of size of effect).

$H_3$: It is expected that the $\delta$
parameter will not be systematically
modulated by the level of affect of items.

\section{Design Plan}

\textbf{Study type:} Observational Study. \\

\textbf{Blinding:} No blinding is involved in this study. \\

\subsection{Study Design}

\emph{Study 1}: All subjects will rate all items
(see Appendix 1) as to the level of affect they
feel with regards to them.

\emph{Study 2}: All participants
indicate their certainty equivalence (CE) for all
combinations of items (10) and certainty levels
(1\%, 5\%, 15\%, 30\%, 50\%, 70\%, 85\%, 95\%, 99\%).
This results in $90$ observations per participant.

\section{Sampling Plan}

\textbf{Existing Data}: Registration prior
to creation of data.

\textbf{Data collection procedures}:
Participants will be recruited through online
channels (e.g. facebook, student groups, etc.).
Participants must be at least 18 years old to
participate. In the first experiments subjects
will be payed 30 DKK for agreeing to participate
in an approx. 10 minute online survey. In the
second experiment subjects will be payed 150 DKK
for agreeing to participate in an approx. 60 minute
online survey.

\textbf{Sample size}:

\emph{Study 1}: 30 participants. \\
\emph{Study 2}: 50 participants.

\textbf{Sample size rationale}:

Power analysis?
Credibility/Density interval 95\%
assuming data generating process?

\section{Variables}

\subsection{Manipulated variables}
\emph{Study 1}: No manipulated variables. \\

\emph{Study 2}: Levels of uncertainty are
manipulated, and are given as
$0.01, 0.05, 0.15, 0.3, 0.5, 0.7, 0.85, 0.95, 0.99$.
Levels of affect differ for each item
(obtained in Study 1).

\subsection{Measured variables}
\emph{Study 1}: The single outcome variable
will be the rating of affect level. This will
be measured on a scale of $0-100$ using a
slider.

\emph{Study 2}: The single outcome variable
is the price that subjects indicate that they
are willing to pay for a ticket in a lottery
(combination of probability of outcome).
This will indicate their certainty equivalence (CE).
This will be measured on a scale of $0-500$ dollars
using a slider. The max is 500 dollars since the
lottery tickets by definition cannot be worth
more than this (see Appendix 2).

\subsection{Indices}

??

\section{Analysis Plan}

All analysis is performed in the programming
language $R$ \autocite{rcore} using $Rstudio$
\autocite{rstudio}. \\

\emph{Study 1}: The affect ratings will be
ordered based on group-level means? \\

\emph{Study 2}: A bayesian generalized nonlinear
mixed effects model is fit to the data using the
$R$ package $brms$ \autocite{brms}.
This is done to estimate the unobserved parameters
$\delta$ and $\gamma$ from the independent variable
probability/uncertainty and the dependent variable
$w(p)$ which is the observed certainty equivalence (CE).
Weakly informative priors are specified for both
$\gamma$ and $\delta$ (see Github).


\section{Discussion}

\printbibliography
\end{document}


