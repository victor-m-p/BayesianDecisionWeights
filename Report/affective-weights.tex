\documentclass[12pt]{article}
%\usepackage{apacite}
\usepackage{wrapfig}
\setlength{\parindent}{0pt}
\usepackage{tgtermes}
\usepackage{setspace}
\usepackage{gensymb}
\doublespacing
\usepackage{graphicx}
\usepackage{float}
\usepackage[utf8]{inputenc}
\usepackage[backend=biber,style=apa,autocite=inline]{biblatex}
\usepackage{fancyhdr}
\pagestyle{fancy}
\lhead{Paper 1}
\rhead{Victor M. Poulsen, Studie Nr.: 201707639}

\DeclareLanguageMapping{english}{english-apa}
\addbibresource{References.bib}
\setcounter{page}{1}

\title{Affective modulation of weighting function}
\author{Victor Møller Poulsen, Studie Nr.: 201707639}

\begin{document}
\maketitle
\leavevmode

\section{Description}
Both Expected-utility theory and prospect theory
posit that humans maximize some version of
utility.
They do so by combination of the values
of two functions \autocite{rottenstreich2001money}.
A value function $v$ transforms objective value to
subjective utility, and a weighting function $w$
distorts probabilities \autocite{rottenstreich2001money,
gonzalez1999shape}. Expected-utility and
prospect theory combine these two paramters in
the simplest way possible
\autocite{rottenstreich2001money}

\[
	\sum w(p_i)v(i),
\]

where $p$ stands for probability and $i$ stands for the
$i^{th}$ gamble.

Prospect Theory \autocite{PT,
tversky1992advances} (PT) is arguably the main model
of human decision making \autocite{
newell2015straight}. It advances theorizing
from expected-utility by postulating that losses and
gains are evaluated as changes in wealth rather
than in regard to end states \autocite{newell2015straight}.

\vspace{3mm}

In \textcite{PT} we find the familiar
(non-linear) S-shaped
value function $v$ which is concave for the gains
domain and convex for losses (where it is steeper as well).
The weight function $w$ is the identity,
$w(p) = p$ in expected-utility theory
\autocite{rottenstreich2001money} whereas
a non-linear probability distortion is proposed in
prospect theory \autocite{PT}. Here $w$
is stylized as being reverse S-shaped,
meaning that it is concave for low probabilities
and convex for high probabilities \textcite{
gonzalez1999shape}. This means that
people underweight changes in probability in
the middle of the spectrum (e.g. $[0.2-0.8]$)
while overweighting changes in probability close
to the end-points (e.g. $[0.0 - 0.2], [0.8 - 1.0]$).
These general characteristics of the weighting
function are empirically well documented
\autocite{tversky1992advances,
wu1996curvature}.

\subsection{Prior work}

There is evidence to support
the notion that the affect of outcomes modulates
the parameters of both $v$
\autocite{hsee2004music} and $w$ \autocite{
rottenstreich2001money}. A main finding is
that the S-shape of the weighting function
$w$ appears to be more pronounced
for high-affect than low-affect outcomes under
uncertainty \autocite{rottenstreich2001money}.
This was shown as a preference reversal in
which a high-affect outcome was preferred for
low probability (1\%) whereas a low-affect
outcome was preferred for high probability (100\%)
\autocite{rottenstreich2001money}.
The finding that affect appears to modulate
both $v$ and $w$ has subsequently been
modelled as an interaction between an
affective system and a deliberative system \textcite{
	mukherjee2010dual,
mukherjee2011thinking}.

\subsection{Focus and parameterization}

In this article we focus exclusively on the
weighting function $w$ while ignoring both
the value function $v$ and the combination
of the two functions. We also restrict ourselves
to the gains domain.
In \textcite{rottenstreich2001money} they
propose that the affective modulation can
be estimated as an affect paramter $a$
in the form:

\[
	w(p) = \frac{p^{1-a}}
	{p^{1-a}+(1-p)^{1-a}}
.\]

where $a \in [0, 1]$ and larger values indicate
greater affect \autocite{rottenstreich2001money}.
The issue with this one-parameter
formulation is that it
does not account for the fact that people
generally show low \emph{eleveation}.
What I mean by that is that the empirically
observed weighting function $w$ typically
crosses the diagonal line
at around $0.3$ rather than $0.5$
\autocite{gonzalez1999shape}. The one-parameter
formulation fixes this point at $0.5$
which can be seen from figure $1$.









The article will
follow much of the methodological strategy
for estimating the parameters of $w$
presented in \textcite{gonzalez1999shape}
in order to validate the affective distortion
hinted at in \textcite{rottenstreich2001money}.
The weighting function $w$
can be parameterized in different ways,
notably with either one or two parameters.
This article follows
\textcite{gonzalez1999shape} by parameterizing
$w$ with two parameters; $\delta$ and $\gamma$.

\vspace{3mm}

The $\delta$ parameter will vary based on
\emph{elevation} (intercept)
\autocite{gonzalez1999shape},
which here simply refers to the overall
perceived attractiveness of outcomes
under uncertainty.

\vspace{3mm}

The $\gamma$ parameter will vary based on
\emph{curvature} (slope)
\autocite{gonzalez1999shape} and is what we
are primarily interested in for our purposes.
It follows as a direct prediction from
\textcite{rottenstreich2001money} that the
curvature ($\gamma$) should be modulated by changes in
the affective level of outcomes.

\vspace{3mm}

The model is:

\[
	\log\frac{w(p)}{1-w(p)} =
	\gamma \log\frac{p}{1-p} + \tau
.\]

where solving for $w(p)$ and setting $\delta = \exp(\tau)$
gives us

\[
	w(p) = \frac{\delta \cdot p^{\gamma}}
	{\delta \cdot p^{\gamma} +
	(1-p)^{\gamma}}
.\]

The above equations are taken from \textcite{
rottenstreich2001money}.

\subsection{Methodology}



but a thorough investigation
of this effect is lacking. This study consits
of two sub-studies. In the first study subjects
will evaluate 10 items on a scale of affect.
In the second study subjects will indicate their
certainty equivalence (CE) as to gambles
involving these questions. Based on this, the
parameters of the weighting function

\[
	w(p) = \frac{\delta \cdot p^{\gamma}}
	{\delta \cdot p^{\gamma} + (1-p)^{\gamma}}
\]

are estimated for each of the 10 items, and it
is calculated whether level of affect (obtained in
study 1) modulates the parameters of the weighting
function. Note that the above is the two-parameter
weighting function
suggested in \autocite{gonzalez1999shape}.

\section{Hypotheses}

$H_1$: It is expected that the 10 questions in
study 1 will - on average - obtain significantly
different ratings as to affective quality.
This is necessary for the follow-up study to
make sense.

$H_2$ It is hypothesized that the $\gamma$
parameter will be higher for items that are
rated as being higher in affect.
(estimate of size of effect).

$H_3$: It is expected that the $\delta$
parameter will not be systematically
modulated by the level of affect of items.

\section{Design Plan}

\textbf{Study type:} Observational Study. \\

\textbf{Blinding:} No blinding is involved in this study. \\

\subsection{Study Design}

\emph{Study 1}: All subjects will rate all items
(see Appendix 1) as to the level of affect they
feel with regards to them.

\emph{Study 2}: All participants
indicate their certainty equivalence (CE) for all
combinations of items (10) and certainty levels
(1\%, 5\%, 15\%, 30\%, 50\%, 70\%, 85\%, 95\%, 99\%).
This results in $90$ observations per participant.

\section{Sampling Plan}

\textbf{Existing Data}: Registration prior
to creation of data.

\textbf{Data collection procedures}:
Participants will be recruited through online
channels (e.g. facebook, student groups, etc.).
Participants must be at least 18 years old to
participate. In the first experiments subjects
will be payed 30 DKK for agreeing to participate
in an approx. 10 minute online survey. In the
second experiment subjects will be payed 150 DKK
for agreeing to participate in an approx. 60 minute
online survey.

\textbf{Sample size}:

\emph{Study 1}: 30 participants. \\
\emph{Study 2}: 50 participants.

\textbf{Sample size rationale}:

Power analysis?
Credibility/Density interval 95\%
assuming data generating process?

\section{Variables}

\subsection{Manipulated variables}
\emph{Study 1}: No manipulated variables. \\

\emph{Study 2}: Levels of uncertainty are
manipulated, and are given as
$0.01, 0.05, 0.15, 0.3, 0.5, 0.7, 0.85, 0.95, 0.99$.
Levels of affect differ for each item
(obtained in Study 1).

\subsection{Measured variables}
\emph{Study 1}: The single outcome variable
will be the rating of affect level. This will
be measured on a scale of $0-100$ using a
slider.

\emph{Study 2}: The single outcome variable
is the price that subjects indicate that they
are willing to pay for a ticket in a lottery
(combination of probability of outcome).
This will indicate their certainty equivalence (CE).
This will be measured on a scale of $0-500$ dollars
using a slider. The max is 500 dollars since the
lottery tickets by definition cannot be worth
more than this (see Appendix 2).

\subsection{Indices}

??

\section{Analysis Plan}

All analysis is performed in the programming
language $R$ \autocite{rcore} using $Rstudio$
\autocite{rstudio}. \\

\emph{Study 1}: The affect ratings will be
ordered based on group-level means? \\

\emph{Study 2}: A bayesian generalized nonlinear
mixed effects model is fit to the data using the
$R$ package $brms$ \autocite{brms}.
This is done to estimate the unobserved parameters
$\delta$ and $\gamma$ from the independent variable
probability/uncertainty and the dependent variable
$w(p)$ which is the observed certainty equivalence (CE).
Weakly informative priors are specified for both
$\gamma$ and $\delta$ (see Github).


\section{Discussion}

\printbibliography
\end{document}


